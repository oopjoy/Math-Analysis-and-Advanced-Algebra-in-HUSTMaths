%必须使用xelatex编译
\documentclass[12pt, a4paper, twoside]{ctexart}%ctexart默认就是宋体 中文小四对应12pt
\usepackage{ctex} %中文宏包
\usepackage{bm}%斜体加粗
\usepackage{CJK}%中文宏包
\usepackage{abstract}%调整摘要的格式
\usepackage{amsmath}%数学宏包
\usepackage{amssymb}%特殊符号宏包
\usepackage{mathrsfs}%mathscr字体宏包
\usepackage{subfigure}
\usepackage[subfigure]{tocloft}%设置目录
\usepackage{titletoc}%设置目录
\usepackage{titlesec}%设置章节标题等内容
\usepackage{amsthm}
%\usepackage{newtxmath} % must come after amsXXX 这个包有点问题,比如会使得公式中的+-号失效
\usepackage{layout}%页面设置宏包
\usepackage[left=2.5cm,right=2.5cm,top=2cm,bottom=2cm]{geometry}%页面设置
\usepackage{fancyhdr}%页眉页脚宏包
\usepackage{color,xcolor}%字的颜色
\usepackage[colorlinks,linkcolor=blue,anchorcolor=green,citecolor=red]{hyperref}%引用并设置颜色
\usepackage{pgffor}
\usepackage{amsfonts}%\mathbb{C}黑板粗体C
\usepackage{multirow}
\usepackage{setspace}%使用spacing环境调节部分行间距
\usepackage{diagbox}
\usepackage{bigstrut}
\usepackage{graphicx}% 插入图片
\usepackage{fontspec} % 字体宏包
\usepackage{fontenc} % 保证英文字体加粗有效

\usepackage{indentfirst}%调节首行缩进


\usepackage{txfonts}
\usepackage{tabularx,tabulary}
\usepackage{array}
\usepackage{bm} %一种加粗的方法
\usepackage{float}
\usepackage{booktabs}
\usepackage{appendix}%附录
\usepackage{newtxtext}
\usepackage{tocbibind}
\usepackage{subfigure}
\usepackage{enumitem}%enumerate扩充包,可调节版式
\usepackage{tikz}%绘制流程图
\usepackage{pgf}
\usetikzlibrary{arrows,shapes,chains}%绘制流程图
\usepackage{listings}%源代码格式
\usepackage{cite}%引用参考文献

\makeatletter%罗马数字 但是大写有问题
\newcommand{\rmnum}[1]{\romannumeral #1}
\newcommand{\Rmnum}[1]{\expandafter@slowromancap\romannumeral #1@}
\makeatother
\newcommand{\Varepsilon}{\mbox{{\Large $\varepsilon$}}}%大号的epsilon字母
\newcommand{\circled}[1]{\pgfmathparse{
		ifthenelse(#1 > 0 && #1 < 21, Hex(9311+#1), Hex(9450)
	}{\fontspec{Cambria}\char"\pgfmathresult}}%生成带圈数字
\newcommand{\rmd}{\mathrm{d}} %快速正体的d


\definecolor{mygreen}{rgb}{0,0.6,0}
\definecolor{mygray}{rgb}{0.5,0.5,0.5}
\definecolor{mymauve}{rgb}{0.58,0,0.82}
\lstset{ %
	backgroundcolor=\color{white},   % choose the background color; you must add \usepackage{color} or \usepackage{xcolor}
	basicstyle=\footnotesize,        % the size of the fonts that are used for the code
	breakatwhitespace=false,         % sets if automatic breaks should only happen at whitespace
	breaklines=true,                 % sets automatic line breaking
	captionpos=bl,                    % sets the caption-position to bottom
	commentstyle=\color{mygreen},    % comment style
	deletekeywords={...},            % if you want to delete keywords from the given language
	escapeinside={\%*}{*)},          % if you want to add LaTeX within your code
	extendedchars=true,              % lets you use non-ASCII characters; for 8-bits encodings only, does not work with UTF-8
	frame=single,                    % adds a frame around the code
	keepspaces=true,                 % keeps spaces in text, useful for keeping indentation of code (possibly needs columns=flexible)
	keywordstyle=\color{blue},       % keyword style
	%language=Python,                 % the language of the code
	morekeywords={*,...},            % if you want to add more keywords to the set
	numbers=left,                    % where to put the line-numbers; possible values are (none, left, right)
	numbersep=5pt,                   % how far the line-numbers are from the code
	numberstyle=\tiny\color{mygray}, % the style that is used for the line-numbers
	rulecolor=\color{black},         % if not set, the frame-color may be changed on line-breaks within not-black text (e.g. comments (green here))
	showspaces=false,                % show spaces everywhere adding particular underscores; it overrides 'showstringspaces'
	showstringspaces=false,          % underline spaces within strings only
	showtabs=false,                  % show tabs within strings adding particular underscores
	stepnumber=1,                    % the step between two line-numbers. If it's 1, each line will be numbered
	stringstyle=\color{orange},     % string literal style
	tabsize=2,                       % sets default tabsize to 2 spaces
	%title=myPython.py                   % show the filename of files included with \lstinputlisting; also try caption instead of title
}

\setmainfont[BoldFont={Times New Roman Bold},AutoFakeBold={Times New Roman Bold}]{Times New Roman}%设置英文和数字的默认字体 
%latex本身的默认Roman Family 字体已经和罗马新体非常相似了但是不完全相同
%全文默认是宋体
\titlespacing*{\section}{0pt}{*0.1}{*1}%调节section的前后行间距, 但参数无法小于1

\newcommand{\bline}{\vspace{0.4cm}\centerline{\rule[0pt]{\linewidth}{0.6pt}} \vspace{0.3cm}}%上下缩进 宽度 粗细

%\hypersetup{colorlinks=true,linkcolor=black,citecolor=blue}

\usepackage{ctexsize,type1cm}

\fancypagestyle{mypage}{
	\fancyhf{}%清空所有设置
	\fancyhead[C]{华\enspace 中\enspace 科\enspace 技\enspace 大\enspace 学}%参数为OC表示只给奇数页加EC表示偶数页
	%leftmark表示当前章标题, rightmark表示小节标题.
	\fancyfoot[C]{\thepage}
	\renewcommand{\headrulewidth}{0.4pt}%改为0pt即可去掉页眉下面的横线
	\renewcommand{\footrulewidth}{0.4pt}%改为0pt即可去掉页脚上面的横线
}
\pagestyle{mypage}
\fancypagestyle{kong}{
	\fancyhf{}%清空所有设置
	\fancyhead[C]{华\enspace 中\enspace 科\enspace 技\enspace 大\enspace 学}
	%\fancyfoot[C]{\thepage} 不要页码的一个页眉页脚
	\renewcommand{\headrulewidth}{0.4pt}%改为0pt即可去掉页眉下面的横线
	\renewcommand{\footrulewidth}{0.4pt}%改为0pt即可去掉页脚上面的横线
}
\setstretch{1.6}%调节全文默认行间距
\setlength{\parindent}{0em}%设置默认首行缩进 2em是默认值.
\begin{document}
\indent%全文首行缩进

\setlength{\abovedisplayskip}{2pt}
\setlength{\belowdisplayskip}{2pt}
\setlength{\abovedisplayshortskip}{2pt}
\setlength{\belowdisplayshortskip}{2pt}
%调整行间公式与上下文的间距
	\begin{spacing}{1.5}
		\begin{center}
			\vspace{1.5cm}{\Huge 华中科技大学\\
				数学与统计学院考研资料\\
				高等代数与高等代数\\}
			\vspace{2cm}{\Large 23届本科生\quad 何七百}
		\end{center}
	\end{spacing}
	\thispagestyle{empty}
	\clearpage
	\section*{序}
{	\setlength{\parindent}{2em}%
本文汇集了2009年到2023年的华中科技大学考研数学分析与高等代数的真题, 按时间顺序排列, 目前仅有题目部分. \par
本人为华中科技大学数学与统计学院2019级本科生, 本科成绩中等, 参与2023届研究生考试, 侥幸考上本院, 数学分析仅得67分, 高等代数121分, 很不均匀, 一方面确实为本人实力不济, 另一方面可能是当年难度差异较大. 总的来说2019年之前的题目难度非常稳定, 但是2019年及之后难度缓慢上升, 特别是数学分析在2022年的题量和难度均加大, 2023年的数学分析更是一举使分数线下调90分之多, 其难度出题类型已经比较接近华五名校. 当然, 某些年份也会偶尔出现难题, 那就大不了大家都不会嘛. 按照目前情况而言, 希望大家主要参考2019年及之后年份的真题. \par 
虽然本人实力不济, 但是这些题目均是参考多方面资料, 经过反复考证. 年份较久的试题答案基本是本人自己所写, 较新的试题网上有较多资料可参考, 本人择优摘录. 只是其中个别题目缺失遗漏, 错误(华科数院不会公布试题, 试题几乎全是回忆版), 本人从其他优秀的资料中, 按照该卷的出题模式, 摘取一些题目进行补足. 当然, 多做十题, 少做十题, 不会影响考研成绩. 数学学习讲究功在平时, 稳固的基本功比独具一格的做法更为重要. 本资料中或多或少会有些瑕疵, 欢迎大家指正 {\footnotesize \footnote{邮箱: hebrave@foxmail.com}}.\par 
本资料仅供学习交流参考, 请勿他用. 希望对诸位考生有所帮助. \par
\begin{flushright}
	第一次编辑于2023年6月9日\\
\end{flushright}
}
\thispagestyle{empty}
\clearpage
%\begin{spacing}{0.1}
%	%\renewcommand{\contentsname}{}
%	\pagenumbering{Roman}
%	\setcounter{page}{1}
%	\tableofcontents
%\end{spacing}
	\clearpage
	\pagenumbering{arabic}
	\setcounter{page}{1}
	\section{数学分析2009}
	%\addcontentsline{toc}{section}{数学分析2009}
	1. 设$G(s,t)$是二元连续可微函数并且满足$aG_s + bG_t\neq 0$, 设$z=f(x,y)$是由$G(cx-az,cy-bz)=0$所确定的函数, 
	其中$a, b, c$是非0的常数. 求$az_x + bz_y$的值. \par
	2. 计算第二型曲线积分
	\[I=\int_C\frac{(x-y)\ \mathrm{d}x+(x+4y)\ \mathrm{d}y}{x^2+4y^2}, \]
	其中$C:x^2+y^2=1$(逆时针方向). \par
	3. 计算三重积分
	\[I=\iiint_\Omega(x+y-z)(x-y+z)(y+z-x)\ \mathrm{dxdydz},\]
	 其中$\Omega=\{(x,y,z)\ |\ 0\leq x+y-z\leq1, 0\leq x-y+z\leq, 0\leq y+z-x\leq1.\}$\par
	4. 请将$f(x)=x(\pi-x)$在$[0,\pi]$上展开为余弦级数并求出在$[-\pi,\pi]$上的和函数. \par 
	5. (1)求$f(x,y,z)=\ln x+2\ln y+3\ln z$在$x^2+y^2+z^2=6R^2$球面上的最大值. \par
	\hspace{1.2em}(2)设$a,b,c$为正数,证明不等式$ab^2c^3\leq108\Big(\dfrac{a+b+c}{6}\Big)^6$. \par
	6. 讨论函数项级数$\sum\limits_{n=0}^{\infty}\dfrac{x^2}{(1+x^2)^n}$在$(-\infty,+\infty)$上的一致收敛性. \par
	7. 设$f(x)$在$(0,1]$上连续可微, 极限$\underset{x\rightarrow0^+}{\mathrm{lim}}x^rf'(x)$存在, 证明$f(x)$在区间$(0,1]$上一致连续. (原题此处仅需证明$r=1/2$的情况)\par
	8. 设正项级数$\sum\limits_{n=1}^{\infty}a_n$发散且$\underset{n\rightarrow\infty}{\mathrm{lim}}\dfrac{a_n}{A_n}=0$, 其中$A_n=\sum\limits_{k=1}^{n}a_k$. 证明: $\sum\limits_{k=1}^{\infty}a_nx^n$的收敛半径为1. \par 
	9. 设$f(x)\in C[0,1]$, 证明$\underset{n\rightarrow\infty}{\mathrm{lim}}\int_0^1nx^{n-1}f(x)\ \rmd x=f(1)$. \par 
	10. 设$f(x)$在$(0,+\infty)$上连续,设广义积分$\int_0^\infty x^a f(x)\ \rmd x$和$\int_{0}^{+\infty} x^b f(x)\ \rmd x$均收敛, 其中$a<b$. 证明: 含参量广义积分$J(y)=\int_{0}^{\infty}x^yf(x)\ \rmd y$在$y\in[a,b]$上一致收敛. 
	 \clearpage
	\section{数学分析2010}
	1. 计算极限$I=\lim\limits_{x\rightarrow+\infty}\big(\sqrt[5]{x^5+x^4}-\sqrt[5]{x^5-x^4}\big)$.\par
	2. 设$Q(x,y)$在$\mathbb{R}^2$上连续可微, 曲线积分$\int_L2xy\ \rmd x+Q(x,y)\ \rmd y$与路径无关, 并且对任意的t都有$\int_{(0,0)}^{(t,1)}2xy\rmd x+Q(x,y)\rmd y
	=\int_{(0,0)}^{(1,t)}2xy\rmd x+Q(x,y)\rmd y$. 求$Q(x,y)$. \par
	3. 设$f(r)$在$\mathbb{R}$上可微且$f(0)=0$, $f'(0)=1$. 求极限\[\
	I=\lim\limits_{r\rightarrow0^+}\frac{1}{r^4}\iiint\limits_{x^2+y^2+z^2\leq r}f\big(\sqrt{x^2+y^2+z^2}\big)\ \rmd x\rmd y\rmd z.\]\par 
	4. 已知$\int_0^\infty e^{-y^2}\ \rmd y=\frac{\sqrt{\pi}}{2}$, 请计算广义含参量积分\[
	I(x)=\int_0^\infty e^{-y^2}\cos (2xy)\ \rmd y\quad x\in\mathbb{R}.\]\par 
	5. 设$a_1>0$, 令$a_{n+1}=\frac{1}{2}(a_n+\frac{1}{a_n})$. 证明: $\{a_n\}$收敛并求出极限.\par
	6. 设广义积分$\int_1^\infty xf(x)\ \rmd x$收敛. 证明广义积分$\int_1^\infty f(x)\ \rmd x$也收敛.\par
	7. 设$f(x)$在$[-1,1]$上二阶连续可微且有$\lim\limits_{x\rightarrow0}\frac{f(x)}{0}=0$, 证明: 数项级数$\sum\limits_{n=1}^{\infty}f(\frac{1}{n})$绝对收敛. \par
	8. 设$\sum\limits_{n=1}^{\infty}u_n(x)$在有界闭区间$[a,b]$上逐点收敛, 其中$u_n(x)$在区间$[a,b]$上可微且存在$M>0$使得$\Big|\sum\limits_{k=1}^{\infty}u_k'(x)\Big|\leq M,\ \forall x\in[a,b],\ \forall n\ge1.\ $证明: $\sum\limits_{n=1}^\infty$在$[a,b]$上一致收敛. \par
	9. 设二元函数$u(t,x)$在$[0,T)\times[a,b]$上二阶连续可微且有$u_t=u_{xx}-u,\ u_x(t,a)=u_x(t,b)=0,\ \forall t\in[0,T)$. 再定义$f(t)=\int_{a}^{b}\Big[u_x^2(t,x)+u^2(t,x)\Big]\rmd x$, 证明: $f(t)$在$[0,T)$上单调递减.\par
	10. 设$f(x)$在$[0,1]$上二阶连续可微, 证明: $\int_{0}^{1}|f'(x)|\ \rmd x\leq 9\int_{0}^{1}|f(x)|+\int_0^1|f''(x)|\ \rmd x.$\par
	\clearpage
	\section{数学分析2011}
	1. 求极限$\lim\limits_{k\rightarrow+\infty}x^\frac{3}{2}(2\sqrt{x}-\sqrt{x+1}-\sqrt{x-1})$.\par 
	2. (存疑)设$f(x)$连续可微且$D:\ x^2+y^2\leq2tx$, 求极限$\lim\limits_{t\rightarrow0^+}\dfrac{\iint_D yf(\sqrt{x^2+y^2})\ \rmd x\rmd y}{t^4}.$\par 
	3. 设曲面$S$是椭球面$z=\sqrt{2(1-x^2-y^2)}$的上半部分, 设$\rho$是原点到椭球面上的任意一点的切平面的距离, 求$\iint_S\dfrac{z}{\rho}\ \rmd S$.\par 
	4. 计算积分$I=\oint_{L^+}y\ \rmd x+z\ \rmd y+x\ \rmd z$, 其中$L^+$为圆周$x^2+y^2+z^2=a^2,\ a>0$及$x+y+z=0$所确定的曲线, 从$z$轴的正向看为逆时针方向.\par
	5. 已知级数$\sum\limits_{n=1}^\infty a_n$收敛. 证明: $\int_{0}^{1}\sum\limits_{n=1}^{\infty}a_nx^n\ \rmd x=\sum\limits_{n=1}^\infty\dfrac{a_n}{n+1}$, 并用之求出级数和$1-\frac{1}{2}+\frac{1}{3}-\frac{1}{4}+\frac{1}{5}+\cdots$.\par 
	6. 计算$\int_0^\infty\dfrac{e^{-ax^2}-e^{-bx^2}}{x}\ \rmd x\quad(b>a>0)$.\par 
	7. 设$a_n>0$, 级数$\sum\limits_{n=1}^\infty a_n$收敛, $r_n=\sum\limits_{k=n}^\infty a_k$. 证明: $\sum\limits_{n=1}^\infty \dfrac{a_n}{r_n}$发散. \par
	8. 设$f(x)$在$[0,2\pi]$上可积, 证明: $\dfrac{1}{2\pi}\int_{0}^{2\pi}f(x)(\pi-x)\ \rmd x=\sum\limits_{n=1}^{\infty}\dfrac{b_n}{n}$.\par
	9. 设$f(x)$在$[0,1]$上二阶连续可微, 证明: $\int_{0}^{1}|f'(x)|\ \rmd x\leq 9\int_{0}^{1}|f(x)|+\int_0^1|f''(x)|\ \rmd x.$\par
	\clearpage
	\section{数学分析2012}
	1. (1)求极限$\lim\limits_{x\rightarrow0}\dfrac{1}{x}\Big(\dfrac{1}{x}-\dfrac{1}{\sin x}\Big)$.\par
	\hspace{1.2em}(2)设$x_1=\sqrt{2},\ x_{n+1}=\sqrt{2x_n}$. 证明: $\{x_n\}$收敛并求出极限. \par
	2. 求下列曲线在第一象限围成的图像的面积: $y=x^2,\ 2y=x^2,\ xy=1,\ xy=2.$\par
	3. 求下列圆环$L$的质量. 已知圆环的形状: $x^2+y^2+z^2=1,\ x+y+z=0$, 并且其线密度为$\rho(x,y,z)=(x-1)^2+(y-1)^2+(z-1)^2.$\par 
	4. 请将$f(x)=|\cos x|$展开为$[-\pi,\pi]$上的Fourier级数. \par
	5. 求幂级数$\sum\limits_{n=0}^{\infty}\dfrac{n+1}{n!}x^n$的收敛域与和函数. \par
	6. 已知$\sum\limits_{n=1}^{\infty}a_n$是发散的正项级数, $S_n$为其部分和, 请用Cauchy收敛原理证明$\sum\limits_{n=1}^{\infty}\dfrac{a_n}{S_n}$发散.\par
	7. 已知$f(x)$在$[0,+\infty)$上连续, 极限$\lim\limits_{x\rightarrow+\infty}f(x)$存在且有限. 证明: $f(x)$在$[0,+\infty)$上有界. \par
	8. 已知反常积分$\int_1^{+\infty}xf(x)\ \rmd x$收敛, 证明: 含参量反常积分$I(y)=\int_1^{+\infty}x^yf(x\,\rmd x)$在$[0,1]$上一致收敛.\par 
	9. 已知$\Omega$为三维空间中的有界区域, $\Omega$的边界为分段光滑的曲面, $\mathbf{n}$为单位外法向量, $u(x,y,z)$为$\Omega$上的二阶连续可偏导函数. 求证:\[
	\iiint_{\Omega}\frac{\partial^2u}{\partial x^2}+\frac{\partial^2u}{\partial y^2}+\frac{\partial^2u}{\partial z^2}\,\rmd x\rmd y\rmd z
	=\iint_{\partial\Omega}\frac{\partial u}{\partial \mathbf{n}}\,\rmd S.\]\par 
	10. 已知$f(x)$在$[0,1]$上二阶连续可微. 证明: $\max\limits_{x\in[0,1]}|f'(x)|\leq|f(1)-f(0)|+\int_0^1|f''(x)|\,\rmd x$. \par 
	\clearpage
	\section{数学分析2013}
	1. (1)求极限$I=\lim\limits_{x\rightarrow0^+}\dfrac{1-\cos x}{\int_0^x\dfrac{\ln (1+xy)}{y}}$.\par 
	\hspace{1.2em}(2)计算含参量广义积分$F(x)=\int_0^{+\infty}\dfrac{\sin(xy)}{ye^y}\,\rmd y.$\par 
	2. 设$R(x,y)$是曲线$C:\,\dfrac{x^2}{a^2}+\dfrac{y^2}{b^2}=1\,(a,b>0)$上的点$(x,y)$处的曲率半径. 计算第一型曲线积分$\int_C R(x,y)\,\rmd S.$\par 
	3. 设$\Omega$是椭球体$C:\,\dfrac{x^2}{a^2}+\dfrac{y^2}{b^2}+\dfrac{z^2}{c^2}\leq1\,(a,b,c>0) $在第一卦限中的部分. 计算三重积分$I=\iiint_\Omega \dfrac{x^2}{a^2}+\dfrac{y^2}{b^2}+\dfrac{z^2}{c^2}\,\rmd x\rmd y\rmd z.$\par
	4. 求幂级数$\sum\limits_{n=1}^\infty(-1)^{n-1}n^2 x^n$的收敛域及其和函数.\par
	5. 在椭球面$C:\,\dfrac{x^2}{a^2}+\dfrac{y^2}{b^2}+\dfrac{z^2}{c^2}=1\,(a,b,c>0) $上确定第一卦限内的一点$(x_0,y_0,z_0)$, 使得过该点的切平面与三个坐标面所围成的四面体的体积最小, 并求出最小体积.\par
	6. 设$f(x)$在$[0,+\infty]$连续, 又设极限$\lim\limits_{x\rightarrow+\infty}f(x)$存在且有限. 证明$f(x)$在$[0,+\infty]$上有界.\par
	7. 设函数$f(x)\mbox{和}g(x)$在$[a,b]$上可导且$g(x)\neq0,\,\forall x\in(a,b)$. 证明: $\exists\ \xi\in(a,b)\ s.t. \dfrac{f(a)-f(\xi)}{g(\xi)-g(b)}=\dfrac{f'(\xi)}{g'(\xi)}.$\par 
	8. 设连续函数列$\{u_n(x)\}$在$[a,b]$上一致收敛于函数$u(x)$. 并且对于$n=1,2,3,\dots,\ u_n(x)\mbox{在}[a,b]$上至少有一个零点. 证明: $u(x)$在$[a,b]$上也至少存在一个零点.\par
	9. 设$f(x)$在$[0,1]$上二阶连续可微且在区间$(0,1)$内存在极值. 证明: $\max\limits_{x\in[0,1]}|f'(x)|\leq\int_0^1|f''(x)|\,\rmd x$.\par 
	10. 证明: $\forall n\in \mathbb{N}^*$, 方程$x^{n+1}+x^n+\cdots+x=1$在$(0,1)$内存在唯一的实根$x_n$, 并证明$\{x_n\}$收敛.\par
	\clearpage
	\section{数学分析2014}
	%\addcontentsline{toc}{section}{数学分析2014}
	1. (1)求极限$\lim\limits_{n\rightarrow\infty}\dfrac{\sum\limits_{k=1}^n k^{\sqrt{2}}}{n^{\sqrt{2}+1}}$.\par
	\hspace{1.2em}(2)设$f(x)$是$[0,\infty)$上的连续函数且$f(x)\ge0$. 用$\Sigma_t$表示抛物面$z=x^2+y^2$介于$z=0\mbox{和}z=t^2(t>0)$之间的部分. 请计算\[
	\lim\limits_{t\rightarrow0^+}\dfrac{\iint\limits_{x^2+y^2\leq t^2}f(x^2+y^2)\,\rmd x\rmd y}{\iint_{\Sigma_t}f(z)\,\rmd S}.\]\par 
	2. 计算第二型曲线积分$\int_\Gamma(x^2-y)\,\rmd x-(x+\sin y^2)\,\rmd y$, 其中$\Gamma$是圆$(x-1)^2+y^2=1$的下装部分, 方向从$(0,0)\mbox{到}(2,0)$.\par 
	3. 计算第二型曲面积分$\iint_{S}x^2\,\rmd y\rmd z+y^2\,\rm z\rmd x+z^2\,\rmd x\rmd z$, 其中$S$为圆锥面$x^2+y^2=z^2$介于平面$z=0\mbox{和}z=1$之间的部分, 方向向下.\par
	4. (1)请将$f(x)=x\sin x,\ x\in[-\pi,\pi]$上展开为Fourier系数, 并判断其收敛性.\par 
	\hspace{1.2em}(2)求$\sum\limits_{n=2}^\infty\dfrac{1}{(n-1)(n+1)}$的和.
	5. 求幂级数$\sum\limits_{n=1}^\infty\dfrac{(-1)^{n-1}x^{2n}}{n(2n-1)}$的收敛域及其和函数.\par
	6. 设$F(u,v)$连续可微, $z=z(x,y)$是由$F(x+\dfrac{z}{y},-y+\dfrac{z}{x})=0$所确定的隐函数. 证明: $x\dfrac{\partial z}{\partial x}+y\dfrac{\partial z}{\partial y}+xy=z$.\par
	7. 设$a_n>0\mbox{且}\sum\limits_{n=1}^\infty a_n$发散, 记$S_n=a_1+a_2+\cdots+a_n$, 证明: 正项级数$\sum\limits_{n=1}^\infty \dfrac{a_n}{S_n^2}$收敛.
	8. 设$f(x)$在$[0,+\infty)$上可微且$e^{-x}f'(x)\mbox{在}[0,+\infty]$上有界. 证明: $e^{-x}f(x)\mbox{在}[0,+\infty)$上有界.\par
	9. (1)设$f(x)\mbox{在}(0,+\infty)$上连续, 又设广义积分$\int_0^{+\infty}f(x)\,\rmd x$收敛且$\lim\limits_{x\rightarrow+\infty}f(x)$存在. 证明: \[\lim\limits_{x\rightarrow+\infty}f(x)=0.\]\par
	\hspace{1.2em}(2)设$f(x)\mbox{在}(0,+\infty)$上连续可微, 又设广义积分$\int_0^{+\infty}f(x)\,\rmd x, \int_0^{+\infty}|f'(x)|\,\rmd x$收敛, 证明: \[|f(X)|\leq\int_0^{+\infty}|f'(t)|\,\rmd t.\]\par 
	
	
	\clearpage
	\section{数学分析2015}
	1. 已知$x_1>0,\ x_{n+1}=\ln (1+x_n)$. 证明: $\{x_n\}$收敛且$\lim\limits_{n\rightarrow\infty}x_n=0$.\par 
	2. 计算第二型曲线积分$I=\int_L\dfrac{x\,\rmd y-y\,\rmd x}{x^2+xy+y^2}$, 其中L是取逆时针方向的圆周$x^2+y^2=1$.\par 
	3. 计算闭曲面$\Big(\dfrac{x^2}{a^2}+\dfrac{y^2}{b^2}+\dfrac{z^2}{c^2}\Big)^2=cx\ (a,b,c>0)$所围的体积.\par
	4. (1)请将$f(x)=x,\ x\in[0,\pi]$展开为余弦级数与正弦级数. \par
	\hspace{1.2em}(2)证明: $\sum\limits_{n=1}^\infty\dfrac{1}{(2n-1)^2}=\dfrac{\pi^2}{8}$.\par 
	5. (1)求$\sum\limits_{n=1}^\infty n^2 x^n$的收敛域.\par
	\hspace{1.2em}(2)求$\sum\limits_{n=1}^\infty\dfrac{n^2+2n}{2^n}$的和.\par
	6. 已知$f(x)\mbox{在}[0,+\infty)$连续可微且$f(0)=0$, $\sum\limits_{n=1}^{\infty}a_n$收敛且$a_n>0$. 证明: $\sum\limits_{n=1}^{\infty}f(a_n)$收敛.\par
	7. 已知$f(x)\mbox{在}(0,+\infty)$二阶可微, $f(0)>0,\ f''(x)<0,\ f'(0)=0$. 证明: $ f(x)\mbox{在}(0,+\infty)$上至少有一根.\par
	8. 已知$f(x)=\int_0^x t^n e^{-t}\,\rmd t,\ \sum\limits_{n=1}^\infty a_n$收敛. 证明: $\sum\limits_{n=1}^\infty\dfrac{a_n}{n!}f(x)\mbox{在}(0,+\infty)$一致收敛. \par
	9. 已知$f(x)\mbox{在}(a,+\infty)$连续, 且$\int_a^{+\infty}f(x)\,\rmd x\mbox{与}\int_0^{+\infty}f'(x)$均收敛. 证明: $\lim\limits_{x\rightarrow+\infty}f(x)=0$.\par 
	10. 设$f(x)\mbox{在}[a,b]$上连续, 在$(a,b)$上可微, 且存在$c\in(a,b)$使得$f(a)f(b)<0,\ f(b)f(c)<0$. 证明: $\exists\ \xi\in(a,b)\ s.t.\ f(\xi)+f'(\xi)=0$.\par 
	
	\clearpage
	\section{数学分析2016}
	1. 已知$f(x,y),\ f_y(x,y)$连续, 且$u(x,y)=\dfrac{1}{2}\int_0^s\,\rmd s\int_{y-x+s}^{y+x-s}f(s,t)\,\rmd t$. 证明: $u_{xx}-u_{yy}=f(x,y)$.\par 
	2. 计算二重积分\[I=\iint_D\dfrac{y}{x}\sin(xy)\,\rmd x\rmd y,\]其中$D:\ xy=\dfrac{\pi}{4},\ xy=\dfrac{\pi}{2},\ y=x,\ y=2x,\ x>0$.\par 
	3. 计算第二型曲面积分\[I=\iint_S x\,\rmd y+y\,\rmd z\rmd x+z\,\rmd x\rmd y,\]其中$S$为$x^2+y^2+z^2=1$的上半部分, 方向向上.\par
	4. 请用闭区间套定理证明零点存在定理. \par
	5. 已知$\lim\limits_{n\rightarrow\infty}n^\frac{5}{4}a_n=0\mbox{且}f(0)=f'(0)=0,\ f(x)\mbox{在}\mathbb{R}$上二阶可微. 证明: $\sum\limits_{n=1}^\infty nf(a_n)$收敛.\par
	6. 求$\sum\limits_{n=0}^\infty\Big(\dfrac{n}{e}\Big)^n\dfrac{(-1)^n}{n!}x^n$的收敛半径和收敛域, 并说明在端点处的收敛情况.\par
	7. 已知$f(\dfrac{a+b}{2})=0$. 证明: $\int_a^b|f(x)|\,\rmd x\leq \dfrac{b-a}{2}\int_a^b|f'(x)|\,\rmd x$.\par 
	8. 已知$f(x)\in C^2(a,b),\ a>0,\ \mbox{且在}(a,b)$上有最大值与最小值. 证明: $\exists\ \xi\in(a,b)\ s.t.\ f'(\xi)-\dfrac{\xi f''(\xi)}{2}=0$.\par 
	9. 请将$f(x)=\mathrm{sgn}\,x,\ x\in[-\pi,\pi]$展开为Fourier级数, 并计算$\sum\limits_{n=1}^\infty\dfrac{(-1)^{n-1}}{2n-1}$.\par 
	10. (存疑)已知$\int_0^1 f(x)\,\rmd x$收敛. 证明: $\int_0^1\sin(x^3)f(x)\, \rmd x$一致收敛.\par
	\clearpage
	\section{数学分析2017}
	1. 计算极限$\lim\limits_{x\rightarrow\infty}x-x^2\ln \Big(1+\dfrac{1}{x}\Big)$.\par 
	2. 计算第二型曲面积分\[\iint_Sx\,\rmd y\rmd z+y\,\rmd z\rmd x+z\,\rmd x\rmd y,\]其中$S$表示曲面$z=\sqrt{1-x^2-y^2}$, 方向向上.\par
	3. 计算$I=\int_0^\infty\dfrac{e^{-ax}-e^{-bx}}{x}\,\rmd x$.\par 
	4. 求$\sum\limits_{n=0}\dfrac{x^{2n}}{2n+1}$的收敛域并求级数和. \par
	5. 请将$f(x)=\arcsin(\cos x),\ x\in[-\pi,\pi]$展开为Fourier级数并计算$\sum\limits_{n=0}^\infty\dfrac{1}{(2n+1)^2}$.\par 
	6. 请用确界原理证明单调有界原理.\par
	7. (导数极限定理)已知$f(x)\mbox{在}(a,b)$上连续可微, 且$\lim\limits_{x\rightarrow a^+}$存在. 证明: $f(x)\mbox{在}x=a$右可导.\par
	8. 已知级数$\sum\limits_{n=0}^\infty a_n$条件收敛, 设$a_n^+=\max\{a_n,0\},\ a_n^-=\max\{-a_n,0\}$. 证明:$
	\lim\limits_{n\rightarrow\infty}\dfrac{\sum\limits_{k=0}^{n}a_k^+}{\sum\limits_{k=0}^{n}a_k^-}=1.$\par 
	9. 已知$f(x,t)$在$\mathbb{R}^{2}$上二阶连续可微且有$f_{xx}(x,t)=f_{tt}(x,t)$. 设$E(t)=\dfrac{1}{2}\int_{t-1}^{1-t}\big[f_x(x,t)\big]^2+\big[f_t(x,t)\big]^2\,\rmd x$. 证明: $E(t)$单调递减.\par
	10. 证明: \[
	\iint_D|f(x,y)-f(0,0)|\,\rmd x\rmd y\leq\iint_D\dfrac{\sqrt{f_x^2(x,y)+f_y^2(x,y)}}{2\sqrt{x^2+y^2}}\,\rmd x\rmd y,\]
	其中$D$表示平面$x^2+y^2\leq1$.\par 

	\clearpage
	\section{数学分析2018}
	1. 计算极限$I=\lim\limits_{x\rightarrow 0}\dfrac{e^x\sin x-x(1+x)}{x^3}$.\par 
	2. 计算二重积分\[
	\iint_D\dfrac{(x+y)^2}{1+(x-y)^2}\,\rmd x\rmd y,\]
	其中$D$为$x+y=-1,\ x+y=1,\ x-y=1,\ x-y=-1$所围成的区域.\par
	3. 计算第二型曲面积分\[
	I=\iint_{S}z^2\,\rmd y\rmd z+y^2\,\rmd z\rmd x-xz\,\rmd x\rmd y,\]
	其中曲面$S$为上半椭球面$\dfrac{(x-1)^2}{4}+\dfrac{(y-2)^2}{9}+\dfrac{(z-3)^2}{16}=1,\ z\ge 3$, 方向向上.\par
	4. (1)设$b>a>0$. 证明: $I(y)=\int_0^{+\infty}e^{-x}\cos(xy)\,\rmd x$关于$y\in[a,b]$一致收敛.\par
	\hspace{1.2em}(2)利用积分次序交换定理, 计算$\int_0^{+\infty}\dfrac{\sin bx-\sin ax}{x}e^{-x}\,\rmd x$.\par
	5. 请将$f(x)=\cos x,\ x\in[0,\pi]$展开为正弦级数并求和函数.\par
	6. 已知数列$\{x_n\}$满足$x_1=\sqrt{2}, x_{n+1}=\sqrt{2+x_n}$. 证明: 数列$\{x_n\}$收敛并给出其极限.\par
	7. 设$p,q$为已知常数, 且$p>0$. 证明: $x^3+px+q=0$有且仅有一个实根.\par
	8. 求幂级数$\sum\limits_{n=1}^\infty\dfrac{\big[3+(-1)^{n-1}\big]^n}{n}x^n$的收敛半径和收敛域.\par
	9. 请用Bolzano-Weierstrass聚点定理证明: 若函数$f(x)$在有界闭区间$[a,b]$上连续, 则它必在此区间上一致连续.\par
	10. 已知$f(x)\mbox{在}[0,\pi]$上连续, 在$(0,\pi)$上可微. 证明: $\exists\ \xi\in(0,\pi)\ s.t.\ f'(\xi)+f(\xi)\cot\xi=0$.\par 
	
	\clearpage
	\section{数学分析2019}
	%\addcontentsline{toc}{section}{数学分析2019}
	1. 已知曲线积分\[\int_{(0,0)}^{(1,t)}(x^2+2xy+y^2)\,\rmd x+Q(x,y)\,\rmd y=-\int_{(0,0)}^{(t,1)}(x^2+2xy+y^2)\,\rmd x+Q(x,y)\,\rmd y,\]
	且积分与路径无关. 求$Q(x,y)$.\par 
	2. 计算二重积分\[\iint_D e^\dfrac{x-y}{x+y}\,\rmd x\rmd y,\]其中区域$D$是$x=0,\ y=0,\ x+y=1$所围成的区域. \par
	3. 计算极限$\lim\limits_{n\rightarrow\infty}\dfrac{1^{\sqrt{2}}+2{\sqrt{2}}+\cdots+n^{\sqrt{2}}}{n^{\sqrt{2}+1}}$.\par
	4. 计算极限$\lim\limits_{x\rightarrow\infty}\Big(x^4-x^2\ln\Big(1+\dfrac{1}{x^2}\Big)\Big)$.\par 
	5. 请将$f(x)=-1,\ x\in[0,\pi]$展开为正弦级数并求$\sum\limits_{k=0}\infty\dfrac{(-1)^k}{2k+1}$.\par 
	6. 计算级数$\sum\limits_{n=1}^\infty\dfrac{1}{n\cdot3^n}$的和.\par
	7. 已知$f(x)\in C(-\infty,0],\ \lim\limits_{x\rightarrow-\infty}f(x)-x=a$. 证明: $f(x)\mbox{在}(-\infty,0]$上一致连续.\par
	8. 已知$\int_0^1x^af(x)$收敛且$J(y)=\int_0^1x^yf(x)\,\rmd x$. 证明: $J(y)\mbox{在}[a,+\infty)$上一致收敛.\par
	9. 设函数$f(x)$二阶可导且在$(0,1)$上有最大值和最小值, 证明: $\exists\ \eta\in(0,1)\ s.t.\ f''(\eta)=f'(\eta)$.\par 
	10. (错误)已知$f(x)>0,\ f(x)$连续. 证明: $f(x)\mbox{在}(a,b)$上单调递增.
	\clearpage
	\section{数学分析2020}
	1. 当$x\in(0,\pi)$时, 计算极限\[I=\lim\limits_{n\infty}\bigg(\dfrac{\sin \dfrac{x}{n}}{n+1}+\dfrac{2\sin \dfrac{2x}{n}}{2n+1}+\cdots\dfrac{n\sin \dfrac{nx}{n}}{n^2+1}
	\bigg).\]\par 
	2. 计算定积分\[I=\int_{0}^{\frac{\pi}{2}}\dfrac{\sin x\cos x}{a^2\sin^2x+b^2\cos x}\,\rmd x.\]\par 
	3. 计算二重积分\[
	\iint_D\dfrac{x}{y^2(1+xy)}\,\rmd x\rmd y,\]
	其中D是$xy=1,\ xy=3,\ y^2=x,\ y^2=3x$所围成的区域.\par
	4. 计算第二型曲线积分\[
	\oint_{L}\dfrac{x\,\rmd y-y\,\rmd x}{4x^2+y^2},\]
	其中L是以$(1,0)$为圆心, 2为半径的圆, 沿逆时针方向.\par
	5. 请将$f(x)=\begin{cases}
		-\dfrac{\pi}{4}&-\pi<x<0\\
		\dfrac{\pi}{4}&0\leq x\leq\pi
	\end{cases}$展开Fourier级数并计算$\sum\limits_{n=1}^\infty\dfrac{1}{(2n-1)^2}$.\par 
	6. 已知$f(x)\mbox{在}[0,+\infty)$上一致连续,$g(x)\mbox{在}[0,+\infty)$上连续, 且$\lim\limits_{x\rightarrow+\infty}[f(x)-g(x)]=0$. 证明: $g(x)\mbox{在}[0,+\infty)$上一致连续.\par 
	7. 已知$f(x)$在$[0,1]$上二阶可微且有$f'(0)=f'(1)=0$. 证明: $\exists\ \xi\in(0,1)\ s.t.\ |f''(\xi)|\ge4|f(1)-f(0)|$.\par 
	8. 已知$f(x)$在$[a,b]$上连续且恒正. 证明: $\int_a^bf(x)\,\rmd x\cdot\int_a^b\dfrac{1}{f(x)}\,\rmd x\ge(b-a)^2$.\par 
	9. 已知$\sum\limits_{n=1}^\infty a_n$绝对收敛. 证明: $\sum\limits_{n=1}^\infty(a_1+a_2+\cdots+a_n)a_n$.\par 
	10. 已知$\sum\limits_{n=1}^\infty a_n\mbox{收敛},\ \{b_n\}$是单调递增的正项数列, 且$b_n\rightarrow+\infty$. 证明:\[
	\lim\limits_{n\rightarrow\infty}\dfrac{a_1b_1+a_2b_2+\cdots a_nb_n}{b_n}=0.\]\par 
	\clearpage
	\section{数学分析2021}
	1. 计算极限$I=\lim\limits_{x\rightarrow+\infty}x\bigg(1+x\ln\Big(1-\dfrac{1}{x}\Big)\bigg)$.\par 
	2. 设$z=z(x,y)$且满足$F(cx-az,cy-bz)=0$. 求$a\dfrac{\partial z}{\partial x}+b\dfrac{\partial z}{\partial y}$.\par 
	3. 已知曲线$x^2+y^2=1,\ x,y>0$. 求该曲线上的一点$(x,y)$使得过该点的切线与x轴, y轴所围成的三角形的面积最小.\par
	4. 计算第二型曲线积分\[\int_L(x+y)^2\,\rmd x-(x^2+y^2)\,\rmd y,\]其中L是由$A(1,1),\ B(2,5),\ C(3,2)$所围成的三角形区域, 沿逆时针方向.\par
	5. 请将$f(x)=x(\pi-x),\ x\in[0,\pi]$展开为余弦级数.\par
	6. 证明: $\int_0^\pi f(x)\,\rmd x=\dfrac{1}{2}\int_0^\pi\big[f(x)+f(\pi-x)\big],\ $并且计算$\int_0^\pi\dfrac{x\sin x}{1+\cos^2x}$.\par 
	7. 证明: $\sum\limits_{n=3}^\infty\ln\Big(1+\dfrac{x}{n\ln^2n}\Big),\ x\in[-1,1]$是一致收敛的.\par
	8. 证明: 广义积分$\int_0^{+\infty}\dfrac{e^{\sin x}\sin 2x}{x}\,\rmd x$收敛.\par
	9. 设$f(x)\mbox{在}[a,b]$可导且$f'(x)$单调递增. 证明: \[
	\int_a^bf(x)\,\rmd x\leq\dfrac{1}{2}(b-a)[f(a)+f(b)].\]\par 
	10. 设$f(x)\mbox{在}[0,1]$上可导且$|f'(x)\leq k<1|$. 若数列$\{x_n\}$满足: $x_0=0,\ x_{n+1}=f(x_n),\ n=0,1,2\cdots$. 证明: $\{x_n\}$收敛. \par
	\clearpage
	\section{数学分析2022}
	1. 给定参数方程$\left\{\begin{aligned}
		&x=\arctan t\\
		&2y-ty^2+e^t=5
	\end{aligned}\right.$. 求$\dfrac{\rmd y}{\rmd x}\Big|_{t=0},\ \dfrac{\rmd^2 y}{\rmd x^2}\Big|_{t=0}$\par
	2. 请将$f(x)=\begin{cases}
		\dfrac{\pi-1}{2}x &0\leq x \leq1\\
		\dfrac{\pi-x}{2} &1\leq x\leq \pi
	\end{cases}$展开为正弦级数.\par
	3. 计算极限$\lim\limits_{n\rightarrow\infty}\dfrac{\sqrt{n}(1^a+2^a+\cdots_n^a)}{\sqrt{1\times3+2\times4+\cdots n(n+2)}}\sin\dfrac{1}{n^a},\ a>0.$\par 
	4. 计算第二型曲面积分\[I=\iint_{S}\dfrac{(x-5)\,\rmd y\rmd z+y\,\rmd x\rmd z+z\,\rmd x\rmd y}{\big[(x-5)^2+y^2+z^2\big]^\frac{3}{2}},\]
	其中S为曲面$(x-5)^2+y^2+z^2=3,\ $方向取外侧. \par
	5. 证明: $f(x)=\int_0^{+\infty}t^{x-1}e^{-t}\ln t\,\rmd t\mbox{在}(0,+\infty)$上连续. \par 
	6. 计算曲面$(x^2+y^2+z^2)^2=4(x^2+y^2-z^2)$所围成的图形的体积.\par
	7. (1)证明: $\sum\limits_{n=1}^{\infty}\dfrac{(-1)^n n(n+1)}{n(n+1)x^2+2^n}\mbox{在}\mathbb{R}$上一致连续.\par
	\hspace{1.2em}(2)计算极限$\lim\limits_{x\rightarrow0}\sum\limits_{n=1}^\infty\dfrac{(-1)^n n(n+1)}{n(n+1)x^2+2^n}$.\par 
	8. 已知$\lim\limits_{n\rightarrow\infty}a_n=a\neq0$. 请用定义证明: $\lim\limits_{n\rightarrow\infty}\dfrac{1}{a_n}=\dfrac{1}{a}$.\par 
	9. 已知$f(x)$为$[A,B]$上的连续函数且有$[a,b]\subseteq[A,B]$, 证明:\[
	\lim\limits_{h\rightarrow0}\dfrac{1}{h}\int_a^b\Big[f(x+h)-f(x)\Big]\,\rmd x=f(b)-f(a).\]\par 
	10. (1)证明: Abel和差变换公式.\par 
	\hspace{1.7em}(2)已知$\lim\limits_{n\rightarrow\infty}\dfrac{1}{n}\sum\limits_{k=1}^n kx_k=A$. 证明: $\lim\limits_{n\rightarrow\infty}\dfrac{1}{n^2}\sum\limits_{k=1}^n k^2x_k=\dfrac{A}{2}$.\par 
	11. 已知广义积分$\int_0^{+\infty}$条件收敛且$f^+=\dfrac{1}{2}(|f|+f),\ f^-=\dfrac{1}{2}(|f|-f)$. 证明: \par
	\hspace{1.7em}(1)$\int_0^{+\infty}f^+\,\rmd x$与$\int_0^{+\infty}f^-\,\rmd x$均发散到$+\infty$.\par 
	\hspace{1.7em}(2)当$A\rightarrow+\infty$时, $\int_0^{A}f^+\,\rmd x$与$\int_0^{A}f^+\,\rmd x$等价.\par
	12. 已知$f(x)$在$[a,b]$上可积且$\lim\limits_{x\rightarrow a^+}f(x)=A$. 设非负的$g(x,h)$满足$\lim\limits_{h\rightarrow 0^+}\int_a^b g(x,h)\,\rmd x=1,\ \forall h\in(a,b)$. 若对于任意的$c\in(a,b)$, 均有当$h\rightarrow 0^+$时, $g(x,h)$关于$x\in[c,b]$一致收敛于0. 即$\forall \varepsilon>0,\ \exists\delta>0,\ $当$0<h<\delta$时, 对$\forall x\in[c,b]$有$|g(x,h)|<\varepsilon$. 证明:\[
	\lim\limits_{h\rightarrow0 ^+}\int_a^bf(x)g(x,h)\,\rmd x=A.\]\par 
	
	\clearpage
	\section{数学分析2023}
	1. 计算极限$\lim\limits_{x\rightarrow0}$$\bigg(-\dfrac{\cot x}{e^{-2x}}+\dfrac{1}{e^{-x}\sin^2x}-\dfrac{1}{x}\bigg).$\par 
	2. 计算广义积分\[
	\int_0^{+\infty}e^{-\frac{1}{4s}}s^{-\frac{3}{2}}e^{-s}\,\rmd s.\]\par 
	3. 判断积分$\int_0^{2\pi}e^{-x^2}\cos x\,\rmd x$的正负, 并证明你的结论.\par
	4. 设$I(x_0,x_1)=\iint_\Sigma\dfrac{e^{-x^\alpha}}{\sqrt{y^2+z^2}}\,\rmd y\rmd z$, 其中$\Sigma$为抛物面$x=y^2+z^2$与平面$x=x_0,\ x=x_1$所围成立体图形的表面的内侧, $\alpha>0,\ x_1>x_0>0$. 求极限$\lim\limits_{\substack{x_0\rightarrow0\\x_1\rightarrow+\infty}}I(x_0, x_1).$\par 
	5. 求解以下问题.\\
	\hspace{1.2em}(1) 证明: 方程$(x+1)^{x+1}=ex^x$有唯一正根.\par
	\hspace{1.2em}(2) 设$p(x)=x-x^2$, 记$[x]$为高斯取整函数, 即不超过$x$的最大整数, 又设$f(x)$是二阶连续可微函数, 对于$\forall k\in\mathbb{N}^*$, 证明: \[
	\int_k^{k+1}f(x)\,\rmd x=\dfrac{f(k+1)+f(k)}{2}-\dfrac{1}{2}\int_k^{k+1}f''(x)p(x-[x])\,\rmd x.\]\par 
	\hspace{1.2em}(3) 若$\beta$为(1)中方程的根, 计算极限$\lim\limits_{n\to \infty}\bigg(\beta+\dfrac{1}{n}\bigg)\bigg(\beta+\dfrac{2}{n}\bigg)\cdots\bigg(\beta+\dfrac{n}{n}\bigg)$.\par 
	6. (1) 对任意的$x>0,\ y\in\mathbb{R}$, 证明: $xy\leq x\ln x-x+e^y$.\par 
	\hspace{1.2em}(2) 设$\alpha,\beta$是任意的非零实数, 对正整数$n$, 证明:\[
	\sum_{k=0}^{n}\binom{\alpha}{k}\binom{\beta}{n-k}=\binom{\alpha+\beta}{n},\]
	\hspace{1.2em}其中$\dbinom{\alpha}{k}=\dfrac{\alpha(\alpha-1)\cdots(\alpha-k+1)}{k!},\ \dbinom{\alpha}{0}=1.$\par 
	7. 设$f: [0,1]\to(0,+\infty)$为连续函数, 常数$a\ge1$. 证明: \[
	\lim\limits_{n\to\infty}\sqrt[n]{\int_0^1(a+x^n)^nf(x)\,\rmd x}=a+1.\]\par 
	8. 设$f(x)$在$\mathbb{R}$上可微, 且对任意的实数$x$有$f(x)=f(x+2k)=f(x+b)$, 其中$k$为正整数, $b$为正无理数. 请用Fourier级数理论证明$f(x)$为常数.\par
	9. 设二元函数$f(x,y)$在$(x_0,y_0)$的某领域$U$上有定义, 且在$U$内有偏导数. 证明: 若偏导数$f_x(x,y)$和$f_y(x,y)$都在点$(x_0,y_0)$可微, 则$f_{xy}(x_0,y_0)=f_{yx}(x_0,y_0).$\par 
	\clearpage
	\section{高等代数2009}
{	\linespread{1.4}\selectfont%对以下内容生效行间距
	%\addcontentsline{toc}{section}{高等代数2009}
	1. 计算行列式(空白处为0)$\begin{vmatrix}
		x & & & & a_n\\
		-1&x & & &a_{n-1}\\
		  &-1&\ddots& &\vdots\\
		  & &\ddots&x&a_2\\
		  & & &-1&x+a_1\\
	\end{vmatrix}$.\par 
	2. 设$A$是$n\times r$阶矩阵, $B$是$r\times s$阶矩阵, 如果$\mathrm{rank}(B)=r$. 证明: \par
	\hspace{1.2em}(1) 如果$AB=0$, 则$A=0$.\par
	\hspace{1.2em}(2) 如果$AB=A$, 则$A=I$.\par
	3. 已知$A=\begin{bmatrix}
		a_{11}&a_{12}&a_{13}\\
		a_{21}&a_{22}&a_{23}\\
	\end{bmatrix}$, $\alpha=(c_1,c_2,c_3)'$, 其中$c_1=\begin{vmatrix}
	a_{12}&a_{13}\\
	a_{22}&a_{23}\\
	\end{vmatrix},\ c_2=\begin{vmatrix}
	a_{13}&a_{11}\\
	a_{23}&a_{21}\\
	\end{vmatrix},\ c_3=\begin{vmatrix}
	a_{11}&a_{12}\\
	a_{21}&a_{22}\\
	\end{vmatrix}$. 证明: \par 
	\hspace{1.2em}(1) $r(A)=2\iff\alpha\neq0.$\par
	\hspace{1.2em}(2) 若$r(A)=2$, 则$\alpha$是$AX=0$的基础解系.\par 
	4. 已知$\alpha_1,\alpha_2,\cdots,\alpha_k$是$AX=0$的一个基础解系, 同时$\beta_1,\beta_2,\cdots,\beta_k$是$AX=0$的一组解. 记$S=(\alpha_1,\alpha_2,\cdots,\alpha_k),\ T=(\beta_1,\beta_2,\cdots,\beta_k)$. 证明: \par 
	\hspace{1.2em}(1) 存在$k$阶方阵$C$使得$T=SC$. \par
	\hspace{1.2em}(2) $\beta_1,\beta_2,\cdots,\beta_k$也是$AX=0$的基础解系$\iff$ $C$可逆.\par
	5. 设$\mathscr{A}$是欧氏空间$E$中的一个线性变换, 对任意的$\alpha,\beta$均有$\left<\mathscr{A}\alpha,\beta\right>=\left<\alpha,\mathscr{A}\beta\right>$. 解决以下问题.\par
	\hspace{1.2em}(1) 证明$\mathscr{A}$的特征值全是实数.\par
	\hspace{1.2em}(2) 问是否可以找到一组基使得$\mathscr{A}$对应对角矩阵.\par
	\hspace{1.2em}(3) 取$E=\mathbb{R}^3$以及取它的一组标准正交基$\{e_1,e_2,e_3\}$, 已知$\mathscr{A}(e_1)=\mathscr{A}(e_2)=\mathscr{A}(e_3)=e_1+e_2+e_3$. 求$\mathbb{R}^3$的另一组标准正交基使得$\mathscr{A}$对应对角矩阵.\par
	6. 设$\mathscr{A}$是$n$维线性空间$V$内的一个幂等线性变换, 即有$\mathscr{A}^2=\mathscr{A}$. \par
	\hspace{1.2em}(1) 证明: $V=\mathrm{Im}\mathscr{A}+\mathrm{ker}\mathscr{A}$, 并说明这是直和吗?\par
	\hspace{1.2em}(2) 设$\mathscr{A}$的秩为$r$. 证明: 存在$V$的一组基使得, 对任意的$\alpha$, 若$\alpha$在这组基下的坐标为$\left(x_1,x_2,\cdots x_n\right)$, 则$\mathscr{A}\alpha$在基下的坐标为$\left(x_1, x_2,\cdots, x_r,0,\cdots,0\right)$.\par 
	\hspace{1.2em}(3) 求出$\mathscr{A}$的最小多项式. \par
	7. 设$A$是$n$阶实对称矩阵, $\alpha$是$n$维实向量, 已知$\begin{bmatrix}
		A&\alpha\\
		\alpha'&1
	\end{bmatrix}$正定.\par
	\hspace{1.2em}(1) 证明: $A$可逆以及判断$A$是否正定.\par
	\hspace{1.2em}(2) 证明: $\alpha'A^{-1}\alpha<1$.\par 
	
	\clearpage
	}
	\section{高等代数2010}
	1. 计算$n$阶行列式 $\begin{vmatrix}
		1-x & x& & & \\
		-1 & 1-x& x& & \\
		& & \ddots& & \\
		& & -1& 1-x& x\\
		& & & -1&1-x\\
	\end{vmatrix}$.\par 
	2. (打洞原理)已知$A,B,C,D$均为$n$阶矩阵且$AC=CA$.\par 
	\hspace{1.2em}(1) 若$A$可逆. 证明: $\begin{vmatrix}
		A&B\\
		C&D\\
	\end{vmatrix}$可逆$\iff \mathrm{det}\left(AD-CB\right)\neq0$.\par 
	\hspace{1.2em}(2) 若$A$不可逆, 请说明上述结论是否仍然成立.\par
	3. 已知矩阵$A=\begin{bmatrix}
		1&-1&0&-1\\
		1&1&2&-1\\
		1&0&1&-1\\
		1&3&4&-1\\
	\end{bmatrix}$. 求一个秩为2的4阶矩阵$B$使得$AB=0$. \par
	4. 设$A$是$n$阶矩阵. 证明: $r(A)=1\iff \exists\,\alpha,\beta\neq \mathbf{0}\ s.t.\ A=\alpha\beta'$.\par 
	5. 设$\mathscr{A}$是一个线性变换, $\lambda$为其对应的一个特征值, $X$是从属于特征值$\lambda$的特征向量. 设$\mathscr{A}'$是线性变换$\mathscr{A}$的转置, $\mu$为其对应的一个特征值且$\lambda\neq\mu$, $Y$是从属于特征值$\lambda$的特征向量. \par
	\hspace{1.2em}(1) 证明: $Y'X=0$.\par 
	\hspace{1.2em}(2) 若$X,Y$均为实向量, 证明: $X,Y$线性无关. \par 
	6. 在$\mathbb{R}^n$中, 已知线性变换$\mathscr{A}$在任意一个基向量$e_i$下的坐标均为$\left(1,1,\cdots,1\right)'$, 其中$e_i$就是第$i$位为1, 其它位为0的向量. \par
	\hspace{1.2em}(1) 求$\mathscr{A}$的特征值与特征向量. \par
	\hspace{1.2em}(2) 求$\mathbb{R}^n$的一组标准正交基使得$\mathscr{A}$在这组基下对应对角矩阵. \par
	7. 已知$A,B$为对称矩阵且$A$为正定矩阵. 证明: 存在矩阵$C$使得$CA+B$为正定矩阵.\par
	8. 已知$\mathscr{A},\mathscr{B}$是复数域线性空间$V$中的两个线性变换且$\mathscr{AB}=\mathscr{BA}$. \par 
	\hspace{1.2em}(1)证明: $\mathscr{A},\mathscr{B}$有一个公共的一维不变子空间. \par
	\hspace{1.2em}(2) 若$\mathscr{A},\mathscr{B}$均可对角化, 证明: 存在一组基使得$\mathscr{A},\mathscr{B}$可同时为对角阵. \par
	
	\clearpage
	\section{高等代数2011}
	1. 计算行列式 $\begin{vmatrix}
		1 & a_1&a_2 &\cdots & a_{n-1}\\
		b_1 & x_1& & &\vdots \\
		b_2& & x_2& & \\
		\vdots& & & \ddots& \\
		b_{n-1}& & & &x_{n-1}\\
	\end{vmatrix}$.\par
	2. 解方程组 $\begin{cases}
		x_1-x_2+x_3+x_4-x_5=0,\\
		x_1+x_2+x_4+x_5=0,\\
		x_1-3x_2+2x_3+x_4-3x_5=0,\\
		3x_1+x_2+x_3+3x_4+x_5=0.\\
	\end{cases}$ \par 
	3. 设$A,B$都是$m\times n$, $C$是$n\times n$矩阵且$A=BC$, $\mathrm{rank}(B)=n$. 证明: $\mathrm{r}(A)=\mathrm{r}(C)$.\par 
	4. 设$\mathscr{A}$是$n$维线性空间$V$上的线性变换且$\mathscr{A}^2=\mathscr{A}$.\par 
	\hspace{1.2em}(1) 证明: $V=\mathrm{Im}\mathscr{A}\oplus\mathrm{ker}\mathscr{A}$.\par 
	\hspace{1.2em}(2) 试求$\mathscr{A}$的最小多项式.\par
	5. 设$A$是$n$实矩阵, $A$的特征值为0或者1. 证明: \par
	\hspace{1.2em}(1) $A$和$A+I$都可逆. \par
	\hspace{1.2em}(2) $A$正交$\iff (A+I)^{-1}+(A'+I)^{-1}=I$. \par 
	6. 设$f(x)$是正的多项式, 即$f(x)>0,\ \forall x$. 又设$A$为实对称矩阵. 证明: \par
	\hspace{1.2em}(1) $f(A)$是正定的.\par
	\hspace{1.2em}(2) $A^2+I$可逆.\par
	7. 设$A$是$m\times n$阶矩阵,$B$是$n\times m$阶矩阵且$m\leq n$. 
	证明: \[\mathrm{det}\left(\lambda E-BA\right)=\lambda^{n-m}\mathrm{det}\left(\lambda E-AB\right).\]\par 
	8. 设$\mathscr{A}$是线性空间$V$上的线性变换, 且$\forall \alpha,\beta\in V$有$\left<\mathscr{A}\alpha,\beta\right>=\left<\alpha,\mathscr{A}\beta\right>$. 证明: \par
	\hspace{1.2em}(1) $\mathrm{Im}\mathscr{A}=\left(\mathrm{ker}\mathscr{A}\right)^\perp$.\par
	\hspace{1.2em}(2) 若$W$是$\mathscr{A}-$子空间(不变子空间), 则$W^\perp$也是$\mathscr{A}-$子空间.\par

	\clearpage
	\section{高等代数2012}
	1. 定义行列式$D=\begin{vmatrix}
		1&1&1&\cdots&1\\
		 &1&1&\cdots&1\\
		 & &1&\cdots&1\\
		 & & &\ddots&\vdots\\
		 & & & &1\\
	\end{vmatrix}$. 求$D$的所有的代数余子式之和.\par
	2. 已知$A$为实矩阵. 证明: $\mathrm{rank}\left(A'A\right)=\mathrm{rank}A=\mathrm{rank}\left(AA'\right)$.\par 
	3. 已知矩阵$P=\begin{bmatrix}
		A&I\\
		I&I\\
	\end{bmatrix}$. 证明: $P$可逆$\iff I-A$可逆, 并且在$(I-A)^{-1}$已知的情况下求$P^{-1}$.\par 
	4. 已知$\mathscr{A,B,C,D}$是$V$上的线性变换且两两可交换, 并有$\mathscr{AC+BD=E}$. 证明: $\mathrm{ker}(\mathscr{AB})=\mathrm{ker}\mathscr{A}+\mathrm{ker}\mathscr{B}$而且是直和.\par
	5. 已知矩阵$A$的元素全部为1.\par
	\hspace{1.2em}(1) 求$A$的特征多项与最小多项式. \par
	\hspace{1.2em}(2) 证明: 矩阵$A$可对角化并求$P$使得$P^{-1}AP$为对角阵.\par
	6. 求正交变换使得$xy+yz+zx=1$为标准方程并指出其曲面类型.\par
	7. 设$A,B$为实对称矩阵. \par
	\hspace{1.2em}(1) 若$A,B$正定且$AB=BA$, 则$AB$也正定.\par
	\hspace{1.2em}(2) 若$A,B$半正定, 则$A+B$也半正定. 若还有$A$正定, 则$A+B$也正定.\par
	8. 设$V$是$\mathbb{R}$上的$2n+1$维线性空间, 而$\mathscr{A,B}$是$V$上的线性变换且$\mathscr{AB}=\mathscr{BA}$. 证明: $\exists\, \lambda,\mu\in\mathbb{R}, v\in V, v\neq0\ s.t.\ \mathscr{A}v=\lambda v, \mathscr{B}v=\mu v$.\par 
	
	\clearpage
	\section{高等代数2013}
	{\linespread{1.35}\selectfont
	1. 设$n$阶行列式$D_n=\begin{vmatrix}
		x&a&\cdots&a\\
		b&x&\cdots&a\\
		& &\ddots& \\
		b&b& &x\\
	\end{vmatrix}$.\par 
	\hspace{1.2em}(1) 计算$D_n$.\par 
	\hspace{1.2em}(2) 计算$\sum\limits_{i=1}^n D_{ni}$, 其中$D_{ni}$表示$(n,i)$元素对应的代数余子式.\par
	2. 已知两个线性方程组$(\mathrm{\rmnum{1}})=\begin{cases}
		x_1-x_2=0\\
		x_2+x_4=0\\
	\end{cases}$. $(\mathrm{\rmnum{2}})=\begin{cases}
	x_1+x_2+x_3=0\\
	x_2+x_3-x_4=0\\
	\end{cases}$.\par 
	\hspace{1.2em}(1) 请分别给$(\mathrm{\rmnum{1}}), (\mathrm{\rmnum{1}})$的一个基础解系.\par
	\hspace{1.2em}(2) 请给出给$(\mathrm{\rmnum{1}}), (\mathrm{\rmnum{1}})$的全部公共解.\par
	3. 设$A,B$是$n$阶幂等矩阵$A^2=A,B^2=B$并且$I-A-B$可逆. 证明: $r(A)=r(B)$.\par 
	4. 设$V$是数域$\mathbb{F}$上的所有$n$阶矩阵所组成的$n\times n$维的线性空间. 给定矩阵$M$, 定义: $\mathscr{A}:\ V\to V,\ \mathscr{A}(A)=MA-AM$.\par 
	\hspace{1.2em}(1) 证明: $\mathscr{A}$是$V$上的线性变换. \par
	\hspace{1.2em}(2) 若$M=\begin{bmatrix}
		d_1& & &\\
		 &d_2& & \\
		 & & \ddots& \\
		 & & &d_n\\
	\end{bmatrix}$ 且$d_i\neq d_j(i\neq j)$, 求$\ker(\mathscr{A})$.\par
	\hspace{1.2em}(3) 若$M=\begin{bmatrix}
		d_1& \\
		&d_2\\
	\end{bmatrix}$, 取$V$的一组基$E_{11}=\begin{bmatrix}
	1&0 \\
	0&0\\
	\end{bmatrix}$, $E_{12}=\begin{bmatrix}
	0&1 \\
	0&0\\
	\end{bmatrix}$, $E_{21}=\begin{bmatrix}
	0&0 \\
	1&0\\
	\end{bmatrix}$, $E_{22}=\begin{bmatrix}
	0&0 \\
	0&1\\
	\end{bmatrix}$. 求$\mathscr{A}$在这组基下对应的矩阵.\par
	5. 设$A=(a_{ij})$是$n$阶方阵, 并且$a_{ij}>0,\ \sum\limits_{k=1}^n a_{ik}=1\,(1\leq i\leq n)$.\par 
	\hspace{1.2em}(1) 证明: 对于$A$的所有特征值$\lambda$均有$\left|\lambda\right|\leq1$并且1是$A$的特征值.\par
	\hspace{1.2em}(2) 证明: 若$A$可逆, 求$A^{-1}$的每行元素之各.\par
	6. 设$n$阶矩阵$A$有$n$个不同的特征值, 且$AB=BA$. 证明: 
	\hspace{1.2em}(1) $A,B$有相同的特征向量. \par
	\hspace{1.2em}(2) $A,B$可对角化. \par
	7. 设$A,B$都是$n$阶实对称矩阵, 并且$A$正定. 证明: 存在可逆阵容$P$使得$P'AP=I$并且$P'BP$是对角阵. \par
	8. (存疑)设$\alpha,\beta,\gamma$是线性空间$\mathbb{R}^n$中的向量, 满足$\alpha+\beta+\gamma=0$. \par
	\hspace{1.2em}(1) 若$\left\langle\alpha,\beta\right\rangle>0$, 证明: $\left<\alpha,\gamma\right><0$, $\left<\beta,\gamma\right><0$且$|r|>\max\{|\alpha|,|\beta|\}$.\par 
	\hspace{1.2em}(2) 若$\left\langle\alpha,\beta\right\rangle<0$, 证明: $\left<\alpha,\beta\right>>\max\{\left<\alpha,\gamma\right>,\left<\beta,\gamma\right>\}$, 并且$|r|>\max\{|\alpha|,|\beta|\}$.\par 
	\hspace{1.2em}(3) 请说明(1)(2)的几何含义.\par
	
	\clearpage
	\section{高等代数2014}
%这一页有点多挤不下.
	1. 设$S_k=\sum\limits_{i=1}^n {a_i}^k\ (k=0,1,2,\cdots,2n-2)$. 计算下列n阶行列式\[
	D_n=\begin{vmatrix}
		S_0&S_1&S_2&\cdots&S_{n-1}\\
		S_1&S_2&S_3&\cdots&S_n\\
		S_2& & & &S_{n+1}\\
		\vdots& & & & \\
		S_{n-1}& & & &S_{2n-2}\\
	\end{vmatrix}.\]
	2. 线性方程组$\begin{cases}
		a_{11}x_1+a_{12}x_2+\cdots+ a_{1n}x_n=0\\
		a_{21}x_1+a_{22}x_2+\cdots+ a_{2n}x_n=0\\
		\cdots\\
		a_{n-1,1}x_1+a_{n-1,2}x_2+\cdots+ a_{n-1,n}x_n=0\\
	\end{cases}$简记为$AX=0$. 设$M_i$是$A$中划去第$i$列后的行列式.\par
	\hspace{1.2em}(1) 证明: $(M_1,-M_2,\cdots,(-1)^{n-1}M_n)'$是此方程组的一个解.\par
	\hspace{1.2em}(2) 如果$r(A=n-1)$, 请给出此方程的一个基础解系.\par
	3. 证明: \par
	\hspace{1.2em}(1) 矩阵方程$AX=0$有解$\iff r(A)=r(A,B)$, 其中$A,B,X$均是n阶矩阵.\par
	\hspace{1.2em}(2) 方程$Ax=0$与$BAx=0$同解$\iff r(A)=r(BA)$, 其中$A,B$是n阶矩阵, $x$是列向量.\par
	4. 设$V$是数域$\mathbb{F}$上的线性空间, 且$V=V_1\oplus V_2$, 其中$V_1,V_2$是$V$的非平凡子空间.\par
	\hspace{1.2em}(1) 证明: 存在唯一的幂等变换$\mathscr{A}$使得$V_1=\{v\in V|\mathscr{A}v=0\}, V_2=\{v\in V|\mathscr{A}v=v\}.$\par
	\hspace{1.2em}(2) 证明: 存在$V$的一组基使得$\mathscr{A}$在这组基下为对角阵.\par
	\hspace{1.2em}(3) 若$V_1$的维数为$d$, 请写出这个对角矩阵.\par
	5. 设$A,B$是复数域上的n阶矩阵且$AB=BA$, 证明: \par
	\hspace{1.2em}(1) $A$的任意的特征子空间都是$B$的不变子空间.\par
	\hspace{1.2em}(2) $A,B$有相同的特征向量.\par
	6. 设$\mathscr{A}$是数域$\mathbb{F}$上的n维线性V上的线性变换. 设$f(\lambda)$是$\mathscr{A}$的特征多项式且$f(x)=g(x)h(x)$, 其中$g(x),h(x)$是互素的. 令$V_1=g(\mathscr{A})V,\ V_2=h(\mathscr{A})V$. 证明: \par
	\hspace{1.2em}(1) $V=V_1\oplus V_2$.\par
	\hspace{1.2em}(2) $\dim V_1=\deg h(x),\ \dim V_2=\deg g(x)$.\par 
	7. 设$A$是n阶正定矩阵. 证明: 存在正交向量组$(v_1,v_2,\cdots,v_n)$使得$A=\sum\limits_{i=1}^n v_i{v_i}'$.\par 
	8. (1) 设V是一个线性空间, 令$\alpha_1,\alpha_2,\cdots,\alpha_m\in V$, 证明: $\alpha_1,\alpha_2,\cdots,\alpha_m$线性无关$\iff$Gram矩阵$G(\alpha_1,\alpha_2,\cdots,\alpha_m)=\left(\left<\alpha_i,\alpha_j\right>\right)_{n\times n}$可逆. \par
	\hspace{1.2em}(2) 设W是V的子空间, $\alpha\in V$, $\alpha'$是$\alpha$在W上的正交投影. 若$\alpha_1,\alpha_2,\cdots,\alpha_m$是W的一组基, 则$\alpha$与$\alpha'$的距离$|\alpha-\alpha'|=\sqrt{\dfrac{\det G(\alpha_1,\alpha_2,\cdots,\alpha_m,\alpha)}{\det G(\alpha_1,\alpha_2,\cdots,\alpha_m)}}$.\par 
}
	\clearpage
	\section{高等代数2015}
	1. 证明$\begin{bmatrix}
		A_{22}&A_{23}&\cdots&A_{2n}\\
		A_{32}&A_{33} & & \\
		\vdots& &\ddots & \\
		A_{n2}& & &A_{nn}\\
	\end{bmatrix}=a_{11} |A|^{n-2}$, 其中$A_{ij}$为$a_{ij}$的代数余子式. \par
	2. (题目缺失)求极大无关组. \par
	3. 请将矩阵$A=\begin{bmatrix}
		1&1&1\\
		1&1&1\\
		1&1&1\\
	\end{bmatrix}$正交标准化. \par
	4. 设$A$为n阶反对称矩阵, $b$是n维列向量, 若$AX=b$有解, 则$r(A)=r\begin{bmatrix}
		A&b\\
		-b'&0\\
	\end{bmatrix}.$\par 
	5. 设$\mathscr{A}$是V上的线性变换, 设$\mathscr{A}$是V上的一般变换, 且$\left<\mathscr{A}(\alpha),\beta\right>+\left<\alpha,\mathscr{B}(\beta)\right>=0,\ \forall\alpha,\beta\in V$. 证明: \par
	\hspace{1.2em}(1) $\mathscr{B}$是V上的线性变换. \par
	\hspace{1.2em}(2) $\mathrm{Im} \mathscr{B}\perp \ker \mathscr{A}$.\par 
	6. 在线性空间V上的线性变换$\mathscr{A}$有互不同的特征值$\lambda_1,\lambda_2,\cdots,\lambda_k$, 分别对应特征向量$\alpha_1,\alpha_2,\cdots,\alpha_k$, 若有$\alpha_1+\alpha_2+\cdots+\alpha_k\in W$且W是$\mathscr{A-}$子空间. 证明: $\dim W\ge k.$\par
	7. 设A是n阶正定矩阵. 证明: $|A|\leq a_{11}a_{22}\cdots a_{nn}$.\par
	8. (1) 设矩阵A,B的特征值均在数域$\mathbb{P}$上, $AB=BA$. 证明: $A,B$有公共的特征向量.\par
	\hspace{1.2em}(2) 设A,B是数域$\mathscr{P}$上的n阶方阵. A,B的特征值均在$\mathbb{P}$中, $AB=BA$, 则存在可逆阵T使得$T^{-1}AT,T^{-1}BT$均为上三角阵.\par
	
	\clearpage
	\section{高等代数2016}
	1. 已知行列式$D=\begin{bmatrix}
		a_{11}&a_{12}&\cdots&a_{1,n-1}&1\\
		a_{21}&a_{22}&\cdots&a_{2,n-1}&1\\
		\vdots& & & &\vdots \\
		a_{n1}&a_{n2}&\cdots&a_{n,n-1}&1\\
	\end{bmatrix}$. 现将$D$的第j行替换为$b_1,b_2,\cdots,b_{n-1},1\,(1\leq j\leq n)$得到了新的行列式$D_j$. 证明: $D=D_1+D_2+\cdots+D_n.$\par 
	2. 设A为n阶正定的实对称矩阵, $y\in\mathbb{R}^n,\ y\neq0$. 证明: $\lim\limits_{m\to \infty}\dfrac{y'A^{m+1}y}{y'A^my}$存在, 
	且正是A的一个特征值.\par
	3. 设$A,B\in\mathbb{P}^{n\times n}$且A的n个特征值均不相同. 证明: A的特征向量也是B的特征向量$\iff$AB=BA.\par
	4. 设V是n维线性空间, $\alpha_1,\alpha_2,\cdots,\alpha_n$是V的一组基. 证明: 任意一组实数$b_1, b_2,\cdots,b_n$恒有一个向量$\alpha\in V$使得$\left<\alpha,\alpha_i\right>=b_i,\,i=1,2,\cdots,n.$\par 
	5. 设$A\in \mathrm{Mn}(\mathbb{K}),\ f_1(x),f_2(x)\in \mathbb{K}[x]$, 记$f(x)=f_1(x)f_2(x)$. 证明: 如果$f_1(x),\ f_2(x)$互素, 那么$f(A)Z=0$的任意一个解都可以唯一的表示为$f_1(A)Z=0$与$f_2(A)Z=0$的一个解的和.\par
	6. 证明: \par 
	\hspace{1.2em}(1) 当A可逆时, 有$\begin{vmatrix}
		A&B\\
		C&D\\
	\end{vmatrix}=\left|A\right|\left|D-CA^{-1}B\right|.$\par
	\hspace{1.2em}(2) 若A不可逆, 则上述结论成立吗?\par 
	7. 设空间$\mathbb{C}^{n\times n}$上的变换$\mathscr{B}_A(X)=XA-AX$, 其中A为复矩阵. 证明: $\mathscr{B}_A(X)$的条件反射至多为$n^2-n.$\par 
	8. (题目缺失)设$\alpha_1,\alpha_2,\cdots,\alpha_n$是一组基. 证明: $\forall b_1,b_2,\cdots,b_n$存在唯一的$\beta_1,\beta_2,\cdots,\beta_n$使得$\left<\beta_i,\alpha_i\right>=b_i.$\par 
	
	\clearpage
	\section{高等代数2017}
	1. 若矩阵$A=\begin{bmatrix}
		3&2&-2\\
		-k&-1&k\\
		4&2&-3\\
	\end{bmatrix}$可对角化, 求k, 并求P使得$P^{-1}AP$为对角矩阵.\par
	2. 设矩阵$A=\begin{bmatrix}
		1&2\\
		-1&3\\
	\end{bmatrix},\ B=\begin{bmatrix}
	2&1\\
	0&4\\
	\end{bmatrix}$. 在空间$\mathbb{P}^{2\times2}$中定义变换$\mathscr{A}(X)=AXB$. 证明: $\mathscr{A}(X)$是线性变换并求出其特征多项式. \par
	3. 设V是Q上的三维线性空间, $T:\ V\to V$是一个线性变换. 已知$\mathscr{A}(x)=y,\mathscr{A}(y)=z,\mathscr{A}(z)=x+y,\ \forall x,y,z\in V$. 证明: 向量$x,y,z$正是V的一组基.\par 
	4. 设A,B是正交. 证明$|\det(A+B)|\leq 2^n$.\par 
	5. 设A,B是n阶矩阵, AB=BA=0, $r(A^2)=r(A)$. 证明: $r(A+B)=r(A)+r(B).$\par 
	6. 设一个矩阵A的前$n-1$个顺序主子式都大于0, 但是$\det(A)=0$. 证明: A是半正定的.\par
	7. 设V为n维线性空间. $w_1,w_2,\cdots,w_r$是V的r个真子空间. 证明: 在V中存在一组标准正交基$\alpha_1,\alpha_2,\cdots,\alpha_n$使得所有$\alpha_i\notin W_1\cup W_2\cup \cdots\cup W_r\,(i=1,2,\cdots,n)$.\par 
	8. 设$\delta$是线性空间V上的线性变换, $W=\{x\in V\big| \delta(x)=2x\}$, 且存在n使得$\delta^n=2^n E$. 证明: W为V的一个子空间, 且$\dim W=\dfrac{\mathrm{tr}(\delta)}{2n}+\dfrac{\mathrm{tr}(\delta^2)}{2^2n}+\cdots+\dfrac{\mathrm{tr}(\delta^n)}{2^n n}$.\par 
	
	\clearpage
	\section{高等代数2018}
	1. 已知线性方程组$\begin{cases}
		3ax_1+(2a+1)x_2+(a+1)x_3=0\\
		(2a-1)x_1+(2a-1)x_2+(a-2)x_3=a+1\\
		(4a-1)x_1+3ax_2+2ax_3=1\\
	\end{cases}$. 当a取何值时些方程组有解, 并求解.\par
	2. 设矩阵$A=\begin{bmatrix}
		-2&0&0\\
		2&a&2\\
		3&1&1\\
	\end{bmatrix}$与$B=\begin{bmatrix}
	-1&0&0\\
	0&2&0\\
	0&0&b\\
	\end{bmatrix}$相似.\par
	\hspace{1.2em}(1) 求$a,b$.\par 
	\hspace{1.2em}(2) 求可逆阵P使得$P^{-1}AP=B$.\par 
	3. 设A是n级实矩阵. 若A的第个元素的代数余子式都等于该元素自身. 证明: A=0或者A是可逆阵.\par
	4. 设A是n级实对称阵. 证明: A可逆$\iff$存在矩阵B使得$AB+B'A$正定.\par
	5. 设A是三阶的实正交矩阵, $|A|=1$. 证明: 存在$f(\lambda)=\lambda^3-t\lambda^2+t\lambda-1\,(-1\leq t\leq 3)$使得$f(A)=0$.\par 
	6. (1) 对于方阵A有$A^2=A$, 则A相似于对角阵.\par
	\hspace{1.2em}(2) 对于n级矩阵A, 若存在$k\in\mathbb{N}$使得$A^k=0$, 则$A^n=0$.\par 
	7. 设$\mathscr{A}$是线性空间V上的线性变换. 证明: $r(\mathscr{A})=r(\mathscr{A}^2)\iff\ker(\mathscr{A})\cap\mathrm{Im}(\mathscr{A})=\{0\}$.\par 
	8. 设A,B是n正定矩阵. 证明: AB正定$\iff$AB=BA. \par 
	
	\clearpage
	\section{高等代数2019}
	1. 设$\alpha,\beta$为n维列向量. 证明: $\left|E+\alpha\beta'\right|=1+\alpha'\beta$.\par 
	2. 设A是$m\times n$型矩阵. 证明: $r(A)=r\iff A=\alpha_1\beta_1+\alpha_2\beta_2+\cdots+\alpha_r\beta_r$, 其中$\alpha_1,\cdots,\alpha_r$是r个m维的线性无关的向量, $\beta_1,\cdots,\beta_r$也是r个m维的线性无关的向量.\par
	3. 设矩阵$A=\left(a_{ij}\right)_{n\times n},\ \forall a_{ij}\in \mathbb{Z}$. 证明: \par
	\hspace{1.2em}(1) 若整数m是A的特征值, 则$m\big| \det(A)$.\par 
	\hspace{1.2em}(2) 若对任意的j都有$\sum\limits_{i=1}^n a_{ij}=m$, 则m是A的特征值. \par
	4. 设$A,B$是n阶实正交矩阵. 证明: \par
	\hspace{1.2em}(1) 若$|A|+|B|=0$, 则$|A+B|=0$.\par 
	\hspace{1.2em}(2) 若n为奇数, 则$|(A-B)(A+B)|=0.$\par 
	5. (题目缺失)设X是$\mathbb{R}^{m\times m}$上的矩阵, $\sigma$是$\mathbb{R}^{m\times m}$上的线性变换.\par 
	6. 设A为实对称矩阵. 定义$\mathrm{sgn}(A)$是矩阵A的符号差(正惯性指数-负惯性指数). 证明: 对于任意的正定矩阵B有$\mathrm{sgn}(A)\leq\mathrm{sgn}(A+B)$.\par
	7. 设实矩阵$A=\begin{bmatrix}
		a_1&b_1& & &\\
		b_1&a_2&b_2& & \\
		   &   &\ddots& & \\
		   &   &   & a_{n-1}&b_{n-1}\\
		   &   &   & b_{n-1}&a_n\\
	\end{bmatrix}$, 其中$b_1,b_2,\cdots,b_{n-1}$均不为0. 证明: \par
	\hspace{1.2em}(1) $r(A)\ge n-1$.\par 
	\hspace{1.2em}(2) A有n个互不相同的特征值.\par
	\hspace{1.2em}(3) 当$n=4,\ a_1=a_2=a_3=a_4=1,\ b_1=b_2=b_3=2$时, 求A的最小多项式. \par
	8. 设V是线性空间, $\sigma,\tau$是V上的线性变换且$\sigma^2=\tau^2=\mathscr{E},\ \sigma\tau+\tau\sigma=\mathscr{O}$. 证明: 存在一组基使得$\sigma,\tau$分别对应矩阵$\begin{bmatrix}
		E&O\\
		O&-E\\
	\end{bmatrix},\ \begin{bmatrix}
	O&E\\
	E&O\\
	\end{bmatrix}$.\par 
	\clearpage 
	\section{高等代数2020}
	1. 计算行列式$\begin{vmatrix}
		a+1&a&a&\cdots& a\\
		a&a+\dfrac{1}{2}&a& & &\\
		a&a&a+\dfrac{1}{3}& & \\
		 & & &\ddots& \\
		 & & & & a+\dfrac{1}{2020}\\
	\end{vmatrix}$ (空上的元素全是$a$).\par 
	2. (题目缺失)关于线性组合. \par
	3. 已知矩阵$A=\begin{bmatrix}
	0&0&\cdots&0&-a_n \\
	1&0&\cdots&0&-a_{n-1}\\
	0&1& & & \\
	 & &\ddots& & \\
	 & & &1&-a_1\\
	\end{bmatrix}$.\par 
	\hspace{1.2em}(1) 求A的特征多项式.\par
	\hspace{1.2em}(2) 证明: A的最小多项式正是其特征多项式.\par
	4. 已知复矩阵A有n个不同的特征值$\lambda_1,\lambda_2,\cdots,\lambda_n$. 证明: 复矩阵B的特征值也为$\lambda_1,\lambda_2,\cdots,\lambda_n\iff$存在$P,Q$使得$A=PQ,\ B=QP$.\par 
	5. 设A,B均为可逆阵. 证明: $A'A=B'B\iff$存在正交阵$C$使得$A=CB$.\par
	6. 已知A为n级实对称矩阵, x为n维列向量, A的正, 负惯性指数为p, q. 设$f(x)=\begin{bmatrix}
		A&x\\
		x'&0\\
	\end{bmatrix}$.\par 
	\hspace{1.2em}(1) 证明: $f(x)=-x'A^*x$.\par 
	\hspace{1.2em}(2) 求$f(X)$的正, 负惯性指数. \par
	7. 设V是数域$\mathbb{F}$上所有n阶矩阵组成的线性空间, 定义线性变换$\mathscr{A}(X)=A'XA\ \forall X\in V$. 证明: $\mathscr{A}$为幂零变换$\iff A$为幂零矩阵.\par
	8. 设A,B是n阶实矩阵且AB=BA. \par
	\hspace{1.2em}(1) 证明: $r(AB)\ge r(A)+r(B)-n$.\par 
	\hspace{1.2em}(2) 若存在$C$和$D$使得$AC+DB=E$, 则$r(AB)=r(A)+r(B)-n$.\par 
	
	\clearpage
	\section{高等代数2021}
	1. 设A为n阶方阵且A的元素全为整数. 证明: $\dfrac{1}{2}$不是A的特征值.\par
	2. 给出方程组$\begin{cases}
		x_1+x_2+x_3+2x_4=3\\
		2x_1+3x_2+(a+1)x_3+7x_4=8\\
		x_1+2x_2+3x_4=3\\
		-x_2+x_3+(a-1)x_4=b-1\\
	\end{cases}$. 当$a,b$满足什么条件时此方程组无解, 有解, 并在有解时给出通解.\par
	3. 设A,B是同阶方阵且AB=BA. 证明: $r(AB)+r(A+B)\leq r(A)+r(B)$.\par 
	4. 设V是数域$\mathbb{F}$上的n维线性空间. $W_1,W_2,\cdots,W_n$是V的n个真子空间. 证明:存在V的一组基$\alpha_1,\alpha_2\cdots,\alpha_n$使得$\alpha_i\notin W_j,\ \forall\,i,j\in\{1,2,\cdots,n\}$.
	5. 设$\sigma$是欧氏空间V上的正交变换, 且其特征值均为实数. 证明: $\left<\sigma(\alpha),\beta\right>=\left<\alpha,\sigma(\beta)\right>$.\par 
	6. 设A为n阶方阵且$\mathrm{tr}(A)=0$. 证明: A相似于一个对角线元素全为0的矩阵. \par
	7. 设A,B为两个同阶实矩阵且A是正定矩阵, B为反对称矩阵. 证明: $\det(A+B)\ge \det A$并指出等号成立的条件. \par
	8. 设A为n阶正定矩阵, $X\in\mathbb{R}^n$是非零列向量. 证明: \par 
	\hspace{1.2em}(1) 矩阵$A+XX'$可逆.\par
	\hspace{1.2em}(2) $0<X'(A+XX')^{-1}X<1$.\par 
	\clearpage
	\section{高等代数2022}
	1. 计算行列式$A=\begin{bmatrix}
		1&a_1&a_2&\cdots&a_{2022}\\
		b_1&1&0& & \\
		b_2&0&2& & \\
		\vdots& & & \ddots& \\
		 b_{2022}& & & &2022\\
	\end{bmatrix}$.\par 
	2. 设矩阵$A=\begin{bmatrix}
		2&2&a\\
		8&2&0\\
		0&0&6\\
	\end{bmatrix}$相似于对角阵$\Gamma$, 求$a$的值并求矩阵P使得$P^{-1}AP=\Gamma$.\par 
	3. 设$A=(a_{ij})$是n阶实矩阵. 已知$a_{ii}>\sum\limits_{j\neq i}|a_{ij}|,\ \forall i=1,2,\cdots,n$. 证明: $|A|>0$.\par 
	4. 设A,B均为n级实对称矩阵, C为n级反对称. 若$A^2+B^2=C^2$, 则$A=B=C=O$.\par 
	5. 设A为$m\times n$阶实矩阵. 证明: $r(E_n-A'A)-r(E_m-AA')=n-m.$\par 
	6. 设A为实对称矩阵且A可逆. 证明: A正定$\iff$任意同级的正矩阵B均有$\mathrm{tr}(AB)>0$.\par 
	7. 设A为n级半正定矩阵且$r(A)=r$. 证明: $V=\{X\in\mathbb{R}^n\big|X'AX=0\}$是$\mathbb{R}^n$的子空间并求出其维数.\par
	8. 设$\mathscr{A}$是欧氏空间V上的一个变换, 定义: 若$\mathscr{A}$有$\left<\mathscr{A}\alpha,\beta\right>+\left<\alpha,\mathscr{A}\beta\right>=0$.
	证明: \par
	\hspace{1.2em}(1) 反称变换是线性变换.\par
	\hspace{1.2em}(2) 若$\mathscr{A}$是反称变换, 则$\mathscr{A-E}$是可逆变换.\par
	\hspace{1.2em}(3) 若$\mathscr{A}$是反称变换, 则$(\mathscr{A+E})(\mathscr{A-E})^{-1}$是正交变换. \par
	\clearpage
	\section{高等代数2023}
	1. 计算行列式$\begin{vmatrix}
			2^2-2& 2^3-2 &\cdots &2^{2023}-2 \\
			3^2-2& 3^3-2 &\cdots &3^{2023}-2 \\
			\vdots &\vdots&\vdots&\vdots\\
			2023^2-2& 2023^3-2 &\cdots &2023^{2023}-2 \\
		\end{vmatrix}.$\par 
	2. 设$A$是$n$阶的幂零矩阵, 求$\mathrm{rank}\begin{bmatrix}
		A^{n-1} & A^{n-2} & \cdots &A &E \\
			& A^{n-1} & A^{n-2}	&\cdots& A\\
			&		& A^{n-1}&\cdots&\vdots\\
			& & &\ddots &A^{n-2}\\
			& & & &A^{n-1}\\
	\end{bmatrix}$
	3. $A$是三阶实矩阵, $A^3=A^2+A+2E$, 设$B=xA^2+yA+E$.\\
	(1) $x,y$满足什么条件(充要条件)时, 矩阵$B$可逆?\\
	(3) 当$x=-1,\ y=1$时, 求尽可能低次数的$f(x)$, 使得$Bf(A)=E$.\par 
	4. 设$A$为$n$阶实对称矩阵, 且方程组$AX=\alpha$无解, $\alpha\in\mathbb{R}^n,\ d>0$. 证明: $n+1$阶的矩阵$\begin{bmatrix}
		A & \alpha\\
		\alpha^T & d\\
	\end{bmatrix}$是不定矩阵(既不是正定, 也不负定). \par 
	5. 设$\sigma$是$n$维线性究竟$V$上的线性变换. 证明: \\
	(1) $\mathrm{Ker}(\sigma)=\mathrm{Ker}(\sigma^2)$当且仅当$\mathrm{Ker}(\sigma)\cap\mathrm{Im}(\sigma)=\{0\}$.\\
	(2) $\mathrm{Im}(\sigma)=\mathrm{Im}(\sigma^2)$当且仅当$V=\mathrm{Ker}(\sigma)+\mathrm{Im}(\sigma).$\par 
	6. 设$\sigma$是线性究竟$\mathbb{R}^3$上的线性变换, 且$\sigma^2=0$. 证明: 存在线性函数$f: \mathbb{R}^3\to \mathbb{R}$和$y\in\mathbb{R}^3$使得对任意的$x\in\mathbb{R}^3$, 有$\sigma(x)=f(x)y$.\par 
	7. 设$n$阶实矩阵$A$的最小多项式为$m(\lambda),\ f(\lambda),\ g(\lambda)$是实系数多项式, $h(\lambda)=[f(\lambda),g(\lambda)]$是二者的最小公倍式, 若$m(\lambda)\,\big|\,h(\lambda)$. 证明: $\mathrm{rank}\big(f(A)\big)+\mathrm{rank}\big(g(A)\big)\ge n.$\par 
	8. 已知$n$维线性空间$V$中存在一族至少有一个不为零变换的线性变换$\sigma_{ij}(1\leq i,j\leq n)$, 且满足$\sigma_{ij}\sigma_{hk}=\begin{cases}
		\sigma_{ik} &j=h\\
		0 &j\neq h
	\end{cases}$. 证明: 存在$V$的一组基$v_1,v_2,\cdots,v_n$, 使得$\sigma_{ij}(v_k)=\begin{cases}
	v_i &j=k\\
	0 &j\neq k\\
	\end{cases}$. \par 
	
	\clearpage

	\begin{thebibliography}{99}
		%\renewcommand{\section}[2]{} 
		%\phantomsection\addcontentsline{toc}{section}{参考文献}
		{\linespread{1.2}\selectfont
			\bibitem{1} 崔尚斌. 数学分析教程(上中下). 科学出版社. %英文名字中~表示空格
			\bibitem{2} 裴礼文. 数学分析中的典型问题与方法(第三版). 高等教育出版社.
			\bibitem{3} 李扬. 数学分析强化讲义.
			\bibitem{4} 谢惠民等. 数学分析习题课讲义(上下). 高等教育出版社. 
			\bibitem{5} 华东师范大学数学系. 数学分析(上下)(第四版). 高等教育出版社.
			\bibitem{6} 李扬. 高等代数强化讲义.
			\bibitem{7} 北京大学数学系前代数小组. 高等代数(第四版). 高等教育出版社. 
			\bibitem{8} 钱吉林. 高等代数题解精粹(第三版). 西北工业大学出版社.
			\bibitem{9} 姚慕生, 谢启鸿. 高等代数(第三版). 复旦大学出版社.
		}
	\end{thebibliography} 
	
	\clearpage
	
\end{document}





